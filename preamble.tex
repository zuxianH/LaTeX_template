% Package------------
\usepackage[left=2.50cm, right=2.50cm, top=2.50cm, bottom=2.50cm]{geometry} %Geomety and margin
%\usepackage{mathptmx} % loads Times New Roman

\usepackage{setspace} 
%\usepackage{helvet} % Text front
\usepackage{amsmath, amsfonts, amssymb, amsthm} % Math、symbols 
\usepackage[english]{babel} %English 
%\usepackage[swedish]{babel} %Swedish
\usepackage{dynkin-diagrams}
\usepackage{graphicx}   % Images 
\usepackage{url}        % URL
\usepackage{bm}         %BF text
\usepackage{multirow}
\usepackage{booktabs}
\usepackage{simpler-wick}
\usepackage{cancel}
\usepackage{algorithm}
\usepackage{algorithmic}
\usepackage{xcolor}
\usepackage{import}
\usepackage{xifthen}
\usepackage{pdfpages}
\usepackage{transparent}
\usepackage{physics}
\usepackage{lastpage}
\usepackage{fancyhdr} %Set header and footer
\pagestyle{fancy}
\lhead{\emph{}} %Name
\chead{} %Title 
\lfoot{}
%\cfoot{Page \thepage\ of \pageref{LastPage}} %If you want..
\rfoot{}
\usepackage{hyperref} %bookmarks
\hypersetup{
	colorlinks = true,
	linkcolor = black,
	urlcolor = red,
	citecolor= black,
	pdftitle = 2}
\usepackage{fancyvrb} % For code
%\begin{Verbatim}[numbers=left, frame=single]
%Hallo world
%\end{Verbatim}
\usepackage{multicol} % Multiple columns
\usepackage{lipsum}  
\usepackage{caption} 
\usepackage{subcaption}
\usepackage{fontawesome5}
\usepackage{physics} % shotcup for physical symbols 
\usepackage{breqn}
\usepackage{enumitem} % item
\usepackage{xcolor}
\usepackage{physics}
\usepackage{breqn}
% MathTheorem
%\newtheorem{theorem}{Theorem}
\theoremstyle{definition}
\newtheorem{theorem}{Theorem}[section]
\newtheorem*{remark}{Remark}
\newtheorem{definition}[theorem]{Definition}
\newtheorem{lemma}{Lemma}[theorem]
\newtheorem{corollary}{Corollary}[theorem]
\newtheorem{example}[theorem]{Example}
\newtheorem{proposition}[theorem]{Proposition}

%_-----------------------------------

\usepackage[framemethod=TikZ]{mdframed}
%theorem
\newcounter{theo}[section]\setcounter{theo}{0}
\renewcommand{\thetheo}{\arabic{section}.\arabic{theo}}
\newenvironment{theo}[2][]{%
	\refstepcounter{theo}%
	\ifstrempty{#1}%
	{\mdfsetup{%
			frametitle={%
				\tikz[baseline=(current bounding box.east),outer sep=0pt]
				\node[anchor=east,rectangle,fill=gray!20]
				{\strut Theorem~\thetheo};}}
	}%
	{\mdfsetup{%
			frametitle={%
				\tikz[baseline=(current bounding box.east),outer sep=0pt]
				\node[anchor=east,rectangle,fill=gray!20]
				{\strut Theorem~\thetheo:~#1};}}%
	}%
	\mdfsetup{innertopmargin=10pt,linecolor=gray!20,%
		linewidth=2pt,topline=true,%
		frametitleaboveskip=\dimexpr-\ht\strutbox\relax
	}
	\begin{mdframed}[]\relax%
		\label{#2}}{\end{mdframed}}

%Lemma
\newcounter{lem}[section] \setcounter{lem}{0}
\renewcommand{\thelem}{\arabic{section}.\arabic{lem}}
\newenvironment{lem}[2][]{%
	\refstepcounter{lem}%
	\ifstrempty{#1}%
	{\mdfsetup{%
			frametitle={%
				\tikz[baseline=(current bounding box.east),outer sep=0pt]
				\node[anchor=east,rectangle,fill=gray!20]
				{\strut Lemma~\thelem};}}
	}%
	{\mdfsetup{%
			frametitle={%
				\tikz[baseline=(current bounding box.east),outer sep=0pt]
				\node[anchor=east,rectangle,fill=gray!20]
				{\strut Lemma~\thelem:~#1};}}%
	}%
	\mdfsetup{innertopmargin=10pt,linecolor=gray!20,%
		linewidth=2pt,topline=true,%
		frametitleaboveskip=\dimexpr-\ht\strutbox\relax
	}
	\begin{mdframed}[]\relax%
		\label{#2}}{\end{mdframed}}


%Definition
%\usepackage{tcolorbox}
%\tcbuselibrary{theorems}
%\newtcbtheorem[number within=section]{Def}{Definition}%
%{colback=white,colframe=black!50, colbacktitle= gray!20,coltitle = black ,fonttitle=\bfseries}{def}
